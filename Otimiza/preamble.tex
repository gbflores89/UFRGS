\documentclass[a4paper,11pt]{article}

\usepackage[T1]{fontenc}
\usepackage{fancyhdr}
\pagestyle{fancy}
\usepackage{ae}
\usepackage[latin1]{inputenc}
\usepackage[brazil]{babel}
\usepackage[brazil]{hyperref}
\usepackage[alf]{abntex2cite}
\usepackage{listings}
 
\usepackage{graphicx}
\usepackage{url}
\usepackage{lastpage}
\usepackage{multirow}
\usepackage{indentfirst}
\usepackage{amsmath}
\usepackage{siunitx}
\usepackage{booktabs}
\usepackage{pgfplots}
\usepackage{tikz}
\usepgfplotslibrary{colormaps} 
\usepackage{xifthen}
%\usepackage[miktex]{gnuplottex}
%\usepackage{gnuplottex}

\usepackage{subfloat}
\usepackage{float} 
\usepackage{subfig}
\usepackage{varwidth}
\newcommand{\subfigref}[1]{\hyperref[#1]{Figura~\ref*{#1}}}

\sisetup{output-decimal-marker = {,}}
\newcommand{\theauthorii}{}
\newcommand{\theauthoriii}{}
\newcommand{\theauthoriv}{}
\renewcommand{\title}[1]{\newcommand{\thetitle}{#1}}
\renewcommand{\author}[1]{\newcommand{\theauthor}{#1}}
\newcommand{\authorii}[1]{\renewcommand{\theauthorii}{#1}}
\newcommand{\authoriii}[1]{\renewcommand{\theauthoriii}{#1}}
\newcommand{\authoriv}[1]{\renewcommand{\theauthoriv}{#1}}

\newcommand{\class}[1]{\newcommand{\theclass}{#1}}
\newcommand{\theclassProffessorii}{}
\newcommand{\theclassProffessoriii}{}
\newcommand{\theclassProffessoriv}{}
\newcommand{\classProffessor}[1]{\newcommand{\theclassProffessor}{#1}}
\newcommand{\classProffessorii}[1]{\renewcommand{\theclassProffessorii}{#1}}
\newcommand{\classProffessoriii}[1]{\renewcommand{\theclassProffessoriii}{#1}}
\newcommand{\classProffessoriv}[1]{\renewcommand{\theclassProffessoriv}{#1}}

\newcommand{\thetitlemin}{}
\newcommand{\titlemin}[1]{\renewcommand{\thetitlemin}{#1}}

\usepackage{palatino}
\renewcommand{\textsc}[1]{\fontshape{sc} \fontfamily{\sfdefault} \selectfont #1}

\mathchardef\period=\mathcode`.
\DeclareMathSymbol{.}{\mathord}{letters}{"3B}
\DeclareMathOperator{\erf}{erf}
\DeclareMathOperator{\diff}{d}

\allowdisplaybreaks
\newenvironment{conditions}[1][]
  {#1 \begin{tabular}[t]{>{$}r<{$} @{${}={}$} l @{$\quad$}r @{  }l}}
  {\end{tabular}\\}

  \usetikzlibrary{shapes,shadows}
  \tikzstyle{exercisebox} = [draw=black, rectangle, 
  inner sep=10pt, style=rounded corners, drop shadow={fill=gray,
  opacity=1}]
  \tikzstyle{exercisetitle} =[fill=white]
 \newcommand{\boxexercise}[2]{ 
    \begin{center}
      \begin{tikzpicture}
        \node [exercisebox, fill=black!1!] (box)
        {\begin{minipage}{0.9\textwidth}
            \setlength{\parindent}{1em}
            \footnotesize  \vspace{\belowdisplayskip} #2
          \end{minipage}};
        \node[exercisetitle, right=10pt, draw=black, style=rounded corners] at
        (box.north west) {Exerc�cio: #1};
      \end{tikzpicture}
    \end{center}
  }

  \tikzstyle{examplebox} = [draw=black, rectangle, 
  inner sep=10pt, style=rounded corners, drop shadow={fill=gray,
  opacity=1}]
  \tikzstyle{exampletitle} =[fill=white]
 \newcommand{\boxexample}[2]{ 
    \begin{center}
      \begin{tikzpicture}
        \node [examplebox, fill=green!10!] (box)
        {\begin{minipage}{0.9\textwidth}
            \setlength{\parindent}{1em}
            \footnotesize \vspace{\belowdisplayskip} #2
          \end{minipage}};
        \node[exampletitle, right=10pt, draw=black, style=rounded corners] at
        (box.north west) {Exemplo: #1};
      \end{tikzpicture}
    \end{center}
  }

\onehalfspacing
\newcommand{\code}[1]{\texttt{#1}}

\renewcommand{\lstlistingname}{C�digo}
\lstset{ language=Matlab,
  basicstyle=\ttfamily,
  basicstyle=\fontfamily{pcr}\fontseries{m}\selectfont\footnotesize,
  breaklines=true,
  columns=fullflexible,
  commentstyle=\color[rgb]{0,0.5,0},
  numbers=left,
  showstringspaces=false,
  morekeywords={matlab2tikz},
  keywordstyle=\color{blue},%
  numberstyle=\tiny\sffamily\color{black},
  frame=tb,
  stringstyle=\color[rgb]{0.5,0,0.5},
  numberstyle=\fontfamily{pcr}\fontseries{m}\selectfont\tiny,
  aboveskip=10pt,belowskip=20pt,
  }

\usepackage{lastpage}
\usepackage{fancyvrb}
% redefine \VerbatimInput
\RecustomVerbatimCommand{\VerbatimInput}{VerbatimInput}%
{fontsize=\scriptsize,frame=single,
rulecolor=\color{lightgray!80},
}

\usepackage{zref-abspage,zref-lastpage}
\makeatletter
  \newcommand*{\iffancylastpage}{%
    \ifnum\zref@extractdefault{LastPage}{abspage}{-1}%
        =\numexpr\value{abspage}+1\relax
      \expandafter\@firstoftwo
    \else
      \expandafter\@secondoftwo
    \fi
  }%
\makeatother

% o novo maketitlepage
\renewcommand{\maketitle}{
% Primeira p�gina de t�tulo  
\pagestyle{empty}
\begin{center}   
\textsc \large  
Universidade Federal do Rio Grande do Sul \\
Escola de Engenharia \\
Departamento de Engenharia Qu�mica \\
Programa de P�s--Gradua��o em Engenharia Qu�mica
\vfill
\Large \textsc \theclass \\
\begin{varwidth}[t]{\textwidth}
Prof:~\theclassProffessor\\
\ifthenelse{\equal{\theclassProffessorii}{}}{}{Prof:~\theclassProffessorii\\}
\ifthenelse{\equal{\theclassProffessoriii}{}}{}{Prof:~\theclassProffessoriii\\}
\ifthenelse{\equal{\theclassProffessoriv}{}}{}{Prof:~\theclassProffessoriv}
\end{varwidth}
\vfill
\huge \bfseries \textsc  
\thetitle
\vfill
\begin{varwidth}[t]{\textwidth}
\Large \bfseries \textsc 
\theauthor\\
\ifthenelse{\equal{\theauthorii}{}}{}{\theauthorii\\}
\ifthenelse{\equal{\theauthoriii}{}}{}{\theauthoriii\\}
\ifthenelse{\equal{\theauthoriv}{}}{}{\theauthoriv}
\end{varwidth}
\vfill
\large \textsc Porto Alegre, RS \\ \today
\end{center}
\clearpage
\setcounter{page}{1}
\pagestyle{fancy}
\lhead{\slshape\bfseries\large\MakeUppercase\thetitlemin}
\rfoot{\iffancylastpage{Provided by \Large \LaTeX}{}}%
} % end maketitle 
