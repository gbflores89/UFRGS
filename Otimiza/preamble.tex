\documentclass[a4paper,11pt]{article}

\usepackage[T1]{fontenc}
\usepackage{fancyhdr}
\pagestyle{fancy}
\usepackage{ae}
\usepackage[latin1]{inputenc}
\usepackage[brazil]{babel}
\usepackage[brazil]{hyperref}
\usepackage[alf]{abntex2cite}
\usepackage{listings}

\usepackage{graphicx}
\usepackage{url}
\usepackage{lastpage}
\usepackage{multirow}
\usepackage{indentfirst}
\usepackage{amsmath}
\usepackage{siunitx}
\usepackage{booktabs}
\usepackage{pgfplots}
\usepackage{tikz}
\usepgfplotslibrary{colormaps} 
\usepackage{xifthen}
% \usepackage[miktex]{gnuplottex}
%\usepackage{gnuplottex}

\usepackage{subfloat}
\usepackage{float} 
\usepackage{subfig}
\usepackage{varwidth}
\newcommand{\subfigref}[1]{\hyperref[#1]{Figura~\ref*{#1}}}

\sisetup{output-decimal-marker = {,}}
\newcommand{\theauthori}{}
\newcommand{\theauthorii}{}
\newcommand{\theauthoriii}{}
\renewcommand{\title}[1]{\newcommand{\thetitle}{#1}}
\renewcommand{\author}[1]{\newcommand{\theauthor}{#1}}
\newcommand{\authori}[1]{\renewcommand{\theauthori}{#1}}
\newcommand{\authorii}[1]{\renewcommand{\theauthorii}{#1}}
\newcommand{\authoriii}[1]{\renewcommand{\theauthoriii}{#1}}
\newcommand{\class}[1]{\newcommand{\theclass}{#1}}
\newcommand{\classProffessor}[1]{\newcommand{\theclassProffessor}{#1}}
\usepackage{palatino}
\renewcommand{\textsc}[1]{\fontshape{sc} \fontfamily{\sfdefault} \selectfont #1}
\mathchardef\period=\mathcode`.
\DeclareMathSymbol{.}{\mathord}{letters}{"3B}

\newcommand{\code}[1]{\texttt{#1}}
\renewcommand{\lstlistingname}{C�digo}
\lstset{ language=Matlab,
  basicstyle=\ttfamily,
  basicstyle=\fontfamily{pcr}\fontseries{m}\selectfont\footnotesize,
  breaklines=true,
  columns=fullflexible,
  commentstyle=\color[rgb]{0,0.5,0},
  numbers=left,
  showstringspaces=false,
  morekeywords={matlab2tikz},
  keywordstyle=\color{blue},%
  numberstyle=\tiny\sffamily\color{black},
  frame=tb,
  stringstyle=\color[rgb]{0.5,0,0.5},
  numberstyle=\fontfamily{pcr}\fontseries{m}\selectfont\tiny,
  aboveskip=10pt,belowskip=20pt
  }
  

\usepackage{fancyvrb}

% redefine \VerbatimInput
\RecustomVerbatimCommand{\VerbatimInput}{VerbatimInput}%
{fontsize=\scriptsize,frame=lines,}




% o novo maketitlepage
\renewcommand{\maketitle}{
% Primeira p�gina de t�tulo  
\pagestyle{empty}
\begin{center}   
\textsc \large  
Universidade Federal do Rio Grande do Sul \\
Escola de Engenharia \\
Departamento de Engenharia Qu�mica \\
Programa de P�s--Gradua��o em Engenharia Qu�mica
\vfill
\Large \textsc \theclass \\ Prof:~\theclassProffessor
\vfill \vfill
\huge \bfseries \textsc  
\thetitle
\vfill \vfill
\begin{varwidth}[t]{\textwidth}
\Large \bfseries \textsc 
\theauthor\\
\ifthenelse{\equal{\theauthori}{}}{}{\theauthori\\}
\ifthenelse{\equal{\theauthorii}{}}{}{\theauthorii\\}
\ifthenelse{\equal{\theauthoriii}{}}{}{\theauthoriii}
\end{varwidth}
\vfill \vfill
\large \textsc Porto Alegre, RS \\ \today
\end{center}
\clearpage
\setcounter{page}{1}
\pagestyle{fancy}
\lhead{}
} % end maketitle 
