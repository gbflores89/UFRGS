
\documentclass[a4paper,11pt]{article}

\usepackage[T1]{fontenc}
\usepackage{fancyhdr}
\pagestyle{fancy}
\usepackage{ae}
\usepackage[latin1]{inputenc}
\usepackage[brazil]{babel}
\usepackage[brazil]{hyperref}
\usepackage[alf]{abntex2cite}

\usepackage{graphicx}
\usepackage{url}
\usepackage{lastpage}
\usepackage{multirow}
\usepackage{indentfirst}
\usepackage{amsmath}
\usepackage{siunitx}
\usepackage{booktabs}
\usepackage{pgfplots}
\usepackage{tikz}
\usepgfplotslibrary{colormaps} 
\usepackage{xifthen}
\usepackage[miktex]{gnuplottex}

\usepackage{subfloat}
\usepackage{float} 
\usepackage{subfig}
\usepackage{varwidth}
\newcommand{\subfigref}[1]{\hyperref[#1]{Figura~\ref*{#1}}}

\sisetup{output-decimal-marker = {,}}
\newcommand{\theauthori}{}
\newcommand{\theauthorii}{}
\newcommand{\theauthoriii}{}
\renewcommand{\title}[1]{\newcommand{\thetitle}{#1}}
\renewcommand{\author}[1]{\newcommand{\theauthor}{#1}}
\newcommand{\authori}[1]{\renewcommand{\theauthori}{#1}}
\newcommand{\authorii}[1]{\renewcommand{\theauthorii}{#1}}
\newcommand{\authoriii}[1]{\renewcommand{\theauthoriii}{#1}}
\newcommand{\class}[1]{\newcommand{\theclass}{#1}}
\newcommand{\classProffessor}[1]{\newcommand{\theclassProffessor}{#1}}
\usepackage{palatino}
\renewcommand{\textsc}[1]{\fontshape{sc} \fontfamily{\sfdefault} \selectfont #1}
\mathchardef\period=\mathcode`.
\DeclareMathSymbol{.}{\mathord}{letters}{"3B}

  
% o novo maketitlepage
\renewcommand{\maketitle}{
% Primeira p�gina de t�tulo  
\pagestyle{empty}
\begin{center}   
\textsc \large  
Universidade Federal do Rio Grande do Sul \\
Escola de Engenharia \\
Departamento de Engenharia Qu�mica \\
Programa de P�s--Gradua��o em Engenharia Qu�mica
\vfill
\Large \textsc \theclass \\ Prof:~\theclassProffessor
\vfill \vfill
\huge \bfseries \textsc  
\thetitle
\vfill \vfill
\begin{varwidth}[t]{\textwidth}
\Large \bfseries \textsc 
\theauthor\\
\ifthenelse{\equal{\theauthori}{}}{}{\theauthori\\}
\ifthenelse{\equal{\theauthorii}{}}{}{\theauthorii\\}
\ifthenelse{\equal{\theauthoriii}{}}{}{\theauthoriii}
\end{varwidth}
\vfill \vfill
\large \textsc Porto Alegre, RS \\ \today
\end{center}
\clearpage
\setcounter{page}{1}
\pagestyle{fancy}
\lhead{}
} % end maketitle 
\begin{document} 
\class{EQP 0026--Otimiza��o de Processos}
\classProffessor{Marcelo Farenzena}
\title{Trabalho 2}
\author{Guilherme Braganholo Fl�res}
% \authori{Aluno2}
% \authorii{Aluno3}
% \authoriii{Aluno4}
 
\maketitle 
 
% % \section*{Defini��es importantes} 
% \begin{equation*} 
% \renewcommand{\arraystretch}{2.5}
% \begin{aligned}
% f(x)& = \text{polinomio em fun��o do vetor }x  \\ \\
% \nabla f(x) &= \left[
% \begin{array}{c}
% \cfrac{\partial f}{\partial x_1}\\
% \cfrac{\partial f}{\partial x_2}\\
% \vdots\\
% \cfrac{\partial f}{\partial x_n}\\
% \end{array}
% \right]\\ \\
% \nabla^2 f(x) &= \left[
% \begin{array}{cccc}
% \cfrac{\partial^2 f}{\partial x_1 \partial x_1} 
% & \cfrac{\partial^2 f}{\partial x_1 \partial x_2} & \cdots
% & \cfrac{\partial^2 f}{\partial x_1 \partial x_n}\\
% \cfrac{\partial^2 f}{\partial x_2 \partial x_1} 
% & \cfrac{\partial^2 f}{\partial x_2 \partial x_2} & \cdots
% & \cfrac{\partial^2 f}{\partial x_2 \partial x_n}\\
% \vdots & \vdots & \ddots
% & \vdots\\
% \cfrac{\partial^2 f}{\partial x_n \partial x_1} 
% & \cfrac{\partial^2 f}{\partial x_n \partial x_2} & \cdots
% & \cfrac{\partial^2 f}{\partial x_n \partial x_n}\\
% \end{array}
% \right]
% \end{aligned}
% \end{equation*}
% \clearpage
% 
\section {Convexidade}
\boxexercise{Convexidade}{
Analisar a convexidade das seguintes fun��es objetivo:
\begin{align}
f_1 (x) &= 2x_1 + 3x_2 +6 \label{eq:1f1}\\ 
f_2 (x) &= x_1^3 \label{eq:1f2}\\
f_3 (x) &= x_1^2 + x_1 x_2 + x_2 + 4 \label{eq:1f3}
\end{align}
}

\subsection*{\autoref{eq:1f1}}
\begin{equation}
f_1 (x) = 2x_1 + 3x_2 +6 \tag{\ref{eq:1f1}}
\end{equation}
Para a \autoref{eq:1f1} temos que o gradiente de $f_1(x)$ � dado por:
\begin{equation}
\nabla f_1(x) = \left[
\begin{array}{c}
2\\
3
\end{array}
\right]
\end{equation}
e para a hessiana temos:
\begin{equation}
\nabla^2 f_1(x) = \left[
\begin{array}{cc}
0& 0\\
0& 0
\end{array}
\right]
\end{equation}

Calculando os autovalores da matriz $\nabla^2 f_1(x)$ temos para $\forall x$:
\begin{itemize}
  \item $\lambda_1 = 0$
  \item $\lambda_2 = 0$
\end{itemize}
portanto temos uma fun��o convexa, neste caso um plano, dado que a fun��o �
linear.


\subsection*{\autoref{eq:1f2}}
\begin{equation}
f_2 (x) = x_1^3 \tag{\ref{eq:1f2}}
\end{equation}
Para a Equa��o \autoref{eq:1f2} temos que o gradiente de $f_2(x)$ � dado por:
\begin{equation}
\nabla f_2(x) = \left[
\begin{array}{c}
3x_1^2
\end{array}
\right]
\end{equation}
e para a hessiana temos:
\begin{equation}
\nabla^2 f_2(x) = \left[
\begin{array}{c}
6x_1
\end{array}
\right]
\end{equation}

Calculando os autovalores da matriz $\nabla^2 f_2(x)$ temos:
\begin{itemize}
  \item $\lambda_1 = \left\{
\begin{array}{ccc}
\geq 0& \rm{se}& x_1 \geq 0\\
\leq 0& \rm{se}& x_1 \leq 0
\end{array}
\right.$
\end{itemize}
portanto teriamos uma fun��o convexa para valores positivos de $x_1$ e n�o
convexa para valores de $x_1$ menores que $0$. Por�m, se observarmos a
\autoref{fig:1f2}-(a), observamos que, em todo o dom�nio da fun��o, n�o �
convexa.
\begin{figure}[h]
\centering
\def\FunctionF(#1){(#1)^3}
\def\FunctionFd(#1){3*(#1)^2}
\def\FunctionFdd(#1){6*(#1)}
\subfloat[$f_1(x)$]{
\begin{tikzpicture}[scale=0.8]
\begin{axis}[
        axis y line=center,
        axis x line=middle, 
        axis on top=true,
    ]
    \addplot [domain=-2:2, samples=51, mark=none, ultra thick, blue] {\FunctionF(x)};
    \node [right, blue] at (axis cs: 0.5,4.5) {$f_2(x)$};
\end{axis}
\end{tikzpicture}}
\hfill
\subfloat[$\nabla f_1(x)$]{
\begin{tikzpicture}[scale=0.8]
\begin{axis}[
        axis y line=center,
        axis x line=middle, 
        axis on top=true,
    ]
    \addplot [domain=-2:2, samples=51, mark=none, ultra thick, orange]
    {\FunctionFd(x)};
	\addplot [domain=-2:2, samples=51, mark=none, ultra thick, red]
	{\FunctionFdd(x)};
\node [right, orange] at (axis cs: 0.5,-5) {$\nabla f_2(x)$};
\node [right, red] at (axis cs: 0.5,-7) {$\nabla^2 f_2(x)$};
\end{axis}
\end{tikzpicture}}
\caption{$f_2(x) = x_1^3$.}
\label{fig:1f2}
\end{figure}

\subsection*{\autoref{eq:1f3}}
\begin{equation}
f_3 (x) = x_1^2 + x_1 x_2 + x_2 + 4  \tag{\ref{eq:1f3}}
\end{equation}
Para a Equa��o \autoref{eq:1f3} temos que o gradiente de $f_3(x)$ � dado por:
\begin{equation}
\nabla f_3(x) = \left[
\begin{array}{c}
2x_1 + x_2 \\
x_1 + 2
\end{array}
\right] 
\end{equation}
e para a hessiana temos:
\begin{equation}
\nabla^2 f_3(x) = \left[
\begin{array}{cc}
2 & 1\\
1 & 0
\end{array}
\right]
\end{equation}

Calculando os autovalores da matriz $\nabla^2 f_3(x)$ temos para $\forall x$:
\begin{itemize}
  \item $\lambda_1 = 1 + \sqrt{2} = 2.41$
  \item $\lambda_2 = 1 - \sqrt{2} = -0.41$
\end{itemize}

Sendo esta matriz n�o definida, n�o � uma fun��o convexa. Pela \autoref{fig:1f3}
tamb�m podemos observar que se trata de uma fun��o do tipo cela.
\begin{figure} [h]
\centering
\begin{tikzpicture}
    \begin{axis}[
        , xlabel = $x_1$
        , ylabel = $x_2$
        , domain = -20:20
        , y domain = -20:20
        , enlargelimits
        , view = {0}{90}
        ]

       \addplot3[samples=100, contour gnuplot={levels={400, 200, 100, 25, 4,
       -25, -50}}, thick] {x^2 + x*y + y + 4};
    \end{axis}
\end{tikzpicture}
\caption{$f_3 (x) = x_1^2 + x_1 x_2 + x_2 + 4$}
\label{fig:1f3}
\end{figure}

\clearpage
\section {Pontos de M�nimo e M�ximo}
\boxexercise{Pontos de M�nimo e M�ximo}{
Analisar pontos de m�ximo e m�nimo das
seguintes fun��es:
\begin{align}
f_1 (x) &= |x| \label{eq:2f1}\\ 
f_2 (x) &= -x_1^4 + x_1^3 +20 \label{eq:2f2}\\
f_3 (x) &= x_1^2 + 2x_1 + 3x_2^2 + 6x_2 + 2 \label{eq:2f3}
\end{align}
}

\subsection*{\autoref{eq:2f1}}
\begin{equation}
f_1 (x) = |x| \tag{\ref{eq:2f1}} 
\end{equation}
Para a \autoref{eq:2f1} temos que o gradiente de $f_1(x)$ � dado por:
\begin{equation}
\nabla f_1(x) = \left[
\begin{array}{cc}
1 &\quad \text{se } x> 0\\
-1 &\quad \text{se } x < 0
\end{array}
\right]
\end{equation}

Apesar de $f_1(x)$ n�o ser possuir gradiente cont�nuo, podemos verificar que
aplicando em $x_1 = 0$ temos $f_1(x) = 0$ e para $\forall$ $x$, $f_1(x) \geq
0$, como podemos observar na \autoref{fig:2f1}-(a):
\begin{figure}[h]
\centering
\def\FunctionF(#1){abs(#1)}
\def\FunctionFd(#1){1}
\def\FunctionFdd(#1){-1}
\subfloat[$f_1(x)$]{
\begin{tikzpicture}[scale=0.8]
\begin{axis}[
        axis y line=center,
        axis x line=middle, 
        axis on top=true,
        xmin=-5,
        xmax=5,
        ymin=0,
        ymax=5,
    ]
    \addplot [domain=-5:5, samples=51, mark=none, ultra thick, blue] {\FunctionF(x)};
    \node [right, blue] at (axis cs: 0.5,4.5) {$f_1(x)$};
\end{axis}
\end{tikzpicture}}
\hfill
\subfloat[$\nabla f_1(x)$]{
\begin{tikzpicture}[scale=0.8]
\begin{axis}[
        axis y line=center,
        axis x line=middle, 
        axis on top=true,
        xmin=-5,
        xmax=5,
        ymin=-1.2,
        ymax=1.2,
    ]
    \addplot [domain=0:5, samples=51, mark=none, ultra thick, orange]
    {\FunctionFd(x)};
	\addplot [domain=-5:0, samples=51, mark=none, ultra thick, orange]
	{\FunctionFdd(x)}; 
\node [right, orange] at (axis cs: 0.5,0.5) {$\nabla f_1(x)$};
\end{axis}
\end{tikzpicture}}
\caption{$f_1(x) = |x|$.}
\label{fig:2f1}
\end{figure}

Portanto podemos concluir que temos um ponto de m�nimo, apesar de n�o ser
poss�vel o calculo da hessiana de $f_1(x)$.

\subsection*{\autoref{eq:2f2}}
\begin{equation}
f_2 (x) = -x_1^4 + x_1^3 +20 \tag{\ref{eq:2f2}} 
\end{equation}
Para a Equa��o \autoref{eq:2f2} temos que o gradiente de $f_2(x)$ � dado por:
\begin{equation}
\nabla f_2(x) = \left[
\begin{array}{c}
-4x_1^3 + 3x_1^2
\end{array}
\right]
\end{equation}
Para um ponto de m�nimo ou m�ximo devemos obter $\nabla f_2(x) = 0$, o que
ocorre quando:
\begin{itemize}
  \item $x_1 = 0$
  \item $x_1 = \cfrac{3}{4}$
\end{itemize}
portanto devemos analizar o valor da hessiana nestes dois pontos, como �
poss�vel observar na \autoref{fig:2f2}.

\begin{figure}[h]
\centering
\def\FunctionF(#1){-(#1)^4 + (#1)^3 +20}
\def\FunctionFd(#1){-4*(#1)^3 + 3*(#1)^2}
\def\FunctionFdd(#1){-12*(#1)^2 + 6*(#1)}

\subfloat[$f_2(x)$]
{\begin{tikzpicture}[scale=0.8]
\begin{axis}[
        axis y line=center,
        axis x line=middle, 
        axis on top=true,
        xmin=-1,
        xmax=1.5,
        ymin=19,
        ymax=21,
    ]
    \addplot [domain=-2:2.5, samples=100, 
    mark=none, ultra thick, blue]{\FunctionF(x)};
\node [right, blue] at (axis cs: 0.5,20.5) {$f_2(x)$};
\end{axis}
\end{tikzpicture}}
\subfloat[$\nabla f_2(x)$ e $\nabla^2 f_2(x)$]
{\begin{tikzpicture}[scale=0.8]
\begin{axis}[
        axis y line=center,
        axis x line=middle, 
        axis on top=true,
        xmin=-1,
        xmax=1,
        ymin=-1,
        ymax=2,
    ]
    \addplot [domain=-2:2.5, samples=100, 
    mark=none, ultra thick, orange]{\FunctionFd(x)};
    \addplot [domain=-2:2.5, samples=100, 
    mark=none, ultra thick, red]{\FunctionFdd(x)};
\node [right, orange] at (axis cs: 0.1,1.5) {$\nabla f_2(x)$};
\node [right, red] at (axis cs: 0.1,1.0) {$\nabla^2 f_2(x)$};
\end{axis}
\end{tikzpicture}}
\caption{$f_2 (x) = -x_1^4 + x_1^3 +20$.}
\label{fig:2f2}
\end{figure}

A hessiana de $f_2(x)$ � dada por:
\begin{equation}
\nabla^2 f_2(x) = \left[
\begin{array}{c}
-12x_1^2 + 6x_1
\end{array}
\right]
\end{equation}

Aplicada em $x_1 = 0$ em $\nabla f_2(x)$ temos:
\begin{equation}
\nabla^2 f_2(0) = \left[
\begin{array}{c}
0
\end{array}
\right]
\end{equation}
onde n�o podemos concluir se � um ponto de m�ximo ou m�nimo, pois � uma matriz
indefinida, o que leva a creer que seja um ponto de inflex�o. Isto pode ser
observando na \autoref{fig:2f2}-(a) onde � apresentado o comportamento da fun��o
$f_2(x)$.

Aplicada em $x_1 = \cfrac{3}{4}$ em $\nabla f_2(x)$ temos:
\begin{equation}
\nabla^2 f_2 \left(\frac{3}{4}\right) = \left[
\begin{array}{c}
-2.25
\end{array}
\right]
\end{equation}
onde podemos concluir diretamente, pelo autovalor �nico $(\lambda_1 = -2.25)$,
que se trata de um ponto de m�ximo.

\subsection*{\autoref{eq:2f3}}
\begin{equation}
f_3 (x) = x_1^2 + 2x_1 + 3x_2^2 + 6x_2 + 2 \tag{\ref{eq:2f3}} 
\end{equation}
Para este �ltimo caso, temos $\nabla f_3(x)$ dado por:
\begin{equation}
\nabla f_3(x) = \left[
\begin{array}{c}
2x_1 + 2\\
6x_2 + 6
\end{array}
\right]
\end{equation}

Para a condi��o necess�ria de m�ximo ou minimos devemos ter $\nabla f_3(x) =
0$, isto ocorre apenas em $x = [-1, -1]^{\rm{T}}$. Partindo da hessiana de
$f_3(x)$ que � igual �:
\begin{equation}
\nabla^2 f_3(x) = \left[
\begin{array}{cc}
2 & 0\\
0& 6
\end{array}
\right]
\end{equation}
e aplicando em $x = [-1, -1]^{\rm{T}}$ temos os uma matriz positiva definida,
com os seguintes autovalores:
\begin{itemize}
  \item $\lambda_1 = 6$
  \item $\lambda_2 = 2$
\end{itemize}
portanto, um ponto de m�nimo, como pode ser visto na \autoref{fig:2f3}.

\begin{figure} [h]
\centering
\begin{tikzpicture}
    \begin{axis}[
        , xlabel = $x_1$
        , ylabel = $x_2$
        , domain = -20:20
        , y domain = -20:20
        , enlargelimits
        , view = {0}{90}
        ]

       \addplot3[samples=100, contour gnuplot={levels={200, 100, 25, 3}},
       thick] {x^2 + 2*x + 3*y^2 + 6*y + 2};
    \end{axis}
\end{tikzpicture}
\caption{$f_3 (x) = x_1^2 + 2x_1 + 3x_2^2 + 6x_2 + 2$}
\label{fig:2f3}
\end{figure}



\section {Otimiza��o sem restri��es}
 
\subsection{Exerc�cio 6.48 \cite{edgar2001optimization}}
\boxexercise{6.48}
{The cost of refined oil when shipped via the Malacca Straits to Japan in
dollars per kiloliter was given (Uchiyama, 1968) as the linear sum of the crude
oil cost, the insurance, customs, freight cost for the oil, loading and
unloading cost, sea berth cost, submarine pipe cost, storage cost, tank area
cost, refining cost, and freight cost of products as:   
\begin{equation}
\label{eq:ehl648}
\begin{aligned}
c = c_c &+ c_i + c_x + \frac{2.09\times 10^{4} t^{-0.3017}}{360} + \frac{1.064
\times 10^{6} at^{0.4925}}{52.47q\left(360\right)} \\
&+ \frac{4.242 \times 10^{4} at^{0.7952} + 1.813 ip \left(nt + 1.2q
\right)^{0.861}}{52.47q\left(360\right)}\\
&+ \frac{4.25 \times 10^{3} a\left(nt+1.2q\right) }{52.47q\left(360\right)}
+ \frac{5.042 \times 10^{3} q^{-0.1899}}{360}\\
&+ \frac{0.1049q^{0.671}}{360}
\end{aligned}
\end{equation}
were: 
\begin{conditions}
a	& annual fixed charges	& {fraction}	&($0.20$) \\
c_c	& crude oil price		&{\$/kL}		&($12.50$)\\
c_i	& insurance cost		&{\$/kL}		&($0.50$)\\
c_x	& customs cost			&{\$/kL}		&($0.90$)\\
i	& interest rate			&				&($0.10$)\\
n	& number of ports		&				&($2$)\\ 
p	& land price			&{\$/m$^2$}		&($7000$)\\
q	& refinery capacity		&{bbl/day}		&\\
t	& tanker size			&{kL} 			&\\
\end{conditions}
Given the values indicated in parentheses, use a computer code to compute the
minimum cost of oil and the optimum tanker size $t$ and refinery size $q$ by
Newton's method and the quasi-Newton method (note that $1~\text{kL} = 6.29
~\text{bbl}$).

The answers in the reference were:
\begin{conditions}
t & $427000$ dwt $\approx$ $485000$ kL \\
q & $185000$ bbl/day
\end{conditions}
}

Substituindo os valores de $a$, $c_c$, $c_i$, $c_x$, $i$, $n$ e $p$ na
\autoref{eq:ehl648} temos:
\begin{equation}
\begin{aligned}
c = 13.9 & + \cfrac{58.055}{{t}^{0.3017}}
+{\cfrac {11.266{t}^{ 0.4925}}{q}}\\
&+{\cfrac {5.294\times10^{-5}\left( 8484{t}^{
0.7952}
+1268.4\left( 2t+ 1.2q \right) ^{ 0.861}\right)}{q}}\\
&+\cfrac{0.630}{{q}^{1.1899}}
+2.914\times10^{-4}{q}^{ 0.671}
\end{aligned}
\end{equation}
onde o gradiente e hessiana de $c$ s�o dados por:
\begin{equation}
\renewcommand{\arraystretch}{2.5}
\nabla c = \left[
\begin{array}{c}
\cfrac{\partial c}{\partial t}\\
\cfrac{\partial c}{\partial q}\\
\end{array}
\right]
\quad \text{e}\quad
\nabla^2 c = \left[
\begin{array}{cc}
\cfrac{\partial^2 c}{\partial t^2}
& \cfrac{\partial^2 c}{\partial t \partial q}\\
\cfrac{\partial^2 c}{\partial q \partial t}
& \cfrac{\partial^2 c}{\partial q^2}
\end{array}
\right]
\end{equation}

As equa��es necess�rias para $\nabla c$ e $\nabla^2c$ s�o dadas por:
\begin{align}
&\begin{aligned}
\cfrac{\partial c}{\partial t}= &- \cfrac{17.515}{{t}^{1.3017}}
+ {\cfrac {5.548}{t^{0.5075}q}}\\
&+ {\cfrac {5.294\times10^{-5} \left(\cfrac{6746.476}{{t}^{0.2048}}
+ \cfrac{2184.185}{\left( 2t+ 1.2q \right) ^{0.139}} \right)}{q}}
\end{aligned}\\
&\begin{aligned}
\cfrac{\partial c}{\partial q}=
& - {\cfrac {11.266{t}^{ 0.4925}}{{q}^{2}}}+ {\cfrac {0.0694 }{\left(
 2t+ 1.2q \right) ^{0.139}q}}\\
&- {\cfrac {5.294\times10^{-5} \left(8484.0{t}^{ 0.7952}
+ 1268.4 \left( 2\,t+ 1.2\,q \right) ^{ 0.861}\right)}{{q}^{2}}}\\
&-\cfrac{0.750}{{q}^{2.1899}} + \cfrac{1.955\times10^{-4}}{{q}^{0.329}}
\end{aligned}\\
&\begin{aligned}
\cfrac{\partial^2 c}{\partial t^2}=& \cfrac{22.800}{{t}^{2.3017}} -
{\cfrac {2.816}{{t}^{1.5075}q}}\\
&+  \cfrac {5.294\times10^{-5} \left( -\cfrac{1381.678}{{t}^{1.2048}}
- \cfrac{607.203}{ \left( 2\,t+ 1.2\,q \right) ^{1.139}}\right) }{q}
\end{aligned}\\
&\begin{aligned}
\cfrac{\partial^2 c}{\partial t \partial q}=& \cfrac{\partial^2 c}{\partial q
\partial t}= {\cfrac {5.548}{{t}^{0.5075}{q}^{2}}} - {\cfrac {0.01929 }{\left(2t+ 1.2q\right)
^{1.139}q}} \\
&- {\cfrac {5.294\times10^{-5} \left(\cfrac{6746.476}{{t}^{0.2048}}
+ \cfrac{2184.184}{\left( 2t+ 1.2q \right) ^{0.139}} \right)}{{q}^{2}}}
\end{aligned}\\
&\begin{aligned}
\cfrac{\partial^2 c}{\partial q^2}=
& {\cfrac {22.531{t}^{ 0.4925}}{{q}^{3}}} - {\cfrac {0.01157}{\left(
2t+ 1.2q \right) ^{1.139}q}} - {\cfrac {0.1387 }{\left( 2t+ 1.2q
\right) ^{0.139}{q}^{2}}}\\
&+ \cfrac {{1.059\times10^{-4}} \left(8484.0{t}^{ 0.7952} +
1268.4\left(2t+1.2q\right)^{0.861} \right)}{{q}^{3}} \\
&+ \cfrac{1.642}{{q}^{3.1899}} -
\cfrac{6.433\times10^{-5}}{{q}^{1.329}}
\end{aligned}
\end{align} 
 
Para a execu��o dos problemas, os C�digos \ref{code:ehl648c1},
\ref{code:ehl648c2} e \ref{code:ehl648c3} foram implemantados em Matlab da
seguinte maneira:
\lstinputlisting[caption={Problema proposto},
label={code:ehl648c1}]{trab2/EHL-6-48/ObjFunc.m} 

\lstinputlisting[caption={Gradiente do problema proposto},
label={code:ehl648c2}]{trab2/EHL-6-48/ObjFuncGrad.m} 

\clearpage
\lstinputlisting[caption={Hessiana do problema proposto},
label={code:ehl648c3}]{trab2/EHL-6-48/ObjFuncHessian.m} 
sendo que:
\begin{conditions}
x(1) & $t$\\
x(2) & $q$
\end{conditions}

\subsubsection{Newton}
Os resultados para o m�todo de Newton foram obtidos com o
\autoref{code:ehl648script1} \lstinputlisting[language=Matlab, caption={Script
de execu��o para a resolu��o do problema utilizando o m�todo de Newton},
label={code:ehl648script1}]{trab2/EHL-6-48/Script1.m}

Os resultados para cada intera��o:
\VerbatimInput[firstline=2, lastline=21]{trab2/EHL-6-48/Resultado1.txt}

Ponto de m�nimo encontrado e fun��o avaliada no ponto �timo:
\begin{equation}
\renewcommand{\arraystretch}{1.2}
x^* = \left[
\begin{array}{c}
502050\\
80850\\ 
\end{array}
\right]
\quad \text{e}\quad
c(x^*)=15.9900
\end{equation}
\VerbatimInput[firstline=29, lastline=38]{trab2/EHL-6-48/Resultado1.txt}
\clearpage

\emph{Output} final do Matlab, confirmando que foi utilizado o m�todo de Newton
\VerbatimInput[firstline=46,
lastline=53]{trab2/EHL-6-48/Resultado1.txt}\bigbreak


Valores de $\nabla c$ (\code{grad}), $\nabla^2 c$ (\code{hessian}) e
$\lambda(\nabla^2 c)$ (\code{eigHessian}) , confirmando que se trata de um
ponto de m�nimo.
\VerbatimInput[firstline=56, lastline=79]{trab2/EHL-6-48/Resultado1.txt}

\clearpage
\subsubsection{quasi-Newton}
Os resultados para um m�todo quasi-Newton foram obtidos com o
\autoref{code:ehl648script2} \lstinputlisting[language=Matlab, caption={Script
de execu��o para a resolu��o do problema utilizando um m�todo quasi-Newton},
label={code:ehl648script2}]{trab2/EHL-6-48/Script2.m}

Os resultados para cada intera��o:
\VerbatimInput[firstline=5, lastline=10]{trab2/EHL-6-48/Resultado2.txt}

Ponto de m�nimo encontrado e fun��o avaliada no ponto �timo:
\begin{equation}
\renewcommand{\arraystretch}{1.2}
x^* = \left[
\begin{array}{c}
500350\\
89660\\ 
\end{array}
\right]
\quad \text{e}\quad
c(x^*)=15.9921
\end{equation}
\VerbatimInput[firstline=20, lastline=29]{trab2/EHL-6-48/Resultado2.txt}

\clearpage
\emph{Output} final do Matlab, confirmando que foi utilizado um m�todo
quasi-Newton
\VerbatimInput[firstline=37,
lastline=44]{trab2/EHL-6-48/Resultado2.txt}\bigbreak


Valores de $\nabla c$ (\code{grad}), $\nabla^2 c$ (\code{hessian}) e
$\lambda(\nabla^2 c)$ (\code{eigHessian}) , confirmando que se trata de um
ponto de m�nimo.
\VerbatimInput[firstline=47, lastline=68]{trab2/EHL-6-48/Resultado2.txt}
% \VerbatimInput[numbers=left]{trab2/EHL-6-48/Resultado2.txt}

\subsubsection{Conlus�o do Exerc�cio 6.48}
Para o problema proposto, o m�todo quasi-Newton apresentou melhor resultado, uma
vez que foi necess�rio um menor n�mero de passos para chegar em $x^*$. Al�m do
mais, o m�todo quasi-Newton n�o necessita do gradiente fa fun��o objetivo.

\clearpage
\subsection{Exerc�cio 6.44 \cite{edgar2001optimization}}
\boxexercise{6.44 (EHL, \citeyear{edgar2001optimization})}{
In a decision problem it is desired to minimize the
expected risk defined as follows: 
\begin{equation}
\label{eq:ehl644}
\varepsilon \{\text{risk} \} = \left(1-P\right)c_1\left[1-F(b) \right] + P c_2
\theta
\left(\frac{b}{2}+\frac{2\pi}{4}\right)F\left(\frac{b}{2}-\frac{\sqrt{2\pi}}{4}\right)
\end{equation}
were: 
\begin{conditions}
F(b) & $\displaystyle\int_{-\infty}^b e^{-u^2 / 2\theta^2} \diff u$ (normal
probability function)\\
c_1 & $1.25\times10^{5}$\\
c_2& $15$\\
\theta & $2000$\\
P &$0.25$
\end{conditions}

Find the minimum expected risk and $b$.  
}

Antes da substitui��o dos valores de $F(b)$, $c_1$, $c_2$, $\theta$ e $P$ na
\autoref{eq:ehl644} devemos encontar uma melhor representa��o para $F(b)$.
\begin{equation}
\label{eq:ehl644f}
F(b) = \int_{-\infty}^b \exp{\left(-\frac{u^2}{2\theta^2}\right)} \diff u
\end{equation}
integrando e aplicando os limites na \autoref{eq:ehl644f} temos:
\begin{equation}
\label{eq:ehl644f2}
F(b) = \sqrt{\dfrac{\pi}{2}} \left(\theta
\erf\left(\dfrac{b}{\theta\sqrt{2}}\right)
+\frac{1}{\sqrt{\dfrac{1}{\theta^2}}}
\right)
\end{equation}
onde a fun��o $\erf$ � tabelada e dada por \cite{Maple10,Matlab2016}:
\begin{equation}
\erf\left(x\right) = \frac{2}{\sqrt{\pi}}\int_0^x \exp{\left(-t^2\right)}\diff t
\end{equation}

Assim, � facilmente obtida a express�o de $\varepsilon\{risk\}$, como
segue:
\newcommand{\auxalpha}{\left(\alpha_{\rm{aux}}\right)}
\newcommand{\auxbeta}{\left(\beta_{\rm{aux}}\right)}
\newcommand{\auxgamma}{\left(\gamma_{\rm{aux}}\right)}
\newcommand{\auxdelta}{\left(\delta_{\rm{aux}}\right)}
\begin{equation}
\label{eq:ehl644-2}
\varepsilon\{risk\} = \left[93750 -
46875 \auxalpha\right]
+3750\auxalpha\auxbeta\auxgamma
\end{equation}
onde $\auxalpha$, $\auxbeta$ e $\auxgamma$ s�o variaveis auxiliares dadas por:
\begin{align}
\auxalpha &= \sqrt{2\pi}\left(2000\erf\left(\frac{b\sqrt{2}}{4000}
\right)+\sqrt{4\times10^6}\right)\\
\auxbeta &= \left(\frac{b+\pi}{2}\right)\\
\auxgamma &= \left(\frac{2b+\sqrt{2\pi}}{4} \right)
\end{align}

A partir da \autoref{eq:ehl644-2} podemos obter os valores de
$\nabla\varepsilon\{risk\}$ e $\nabla^2\varepsilon\{risk\}$ como seguem:
\begin{align}
&\begin{aligned}
\cfrac{\partial \varepsilon\{risk\}}{\partial b}= &
-93750 \exp\auxdelta 
+ 1875 \auxalpha \auxgamma\\
&+7500\auxbeta \auxgamma \exp\auxdelta\\
&+1875\auxalpha\auxbeta
\end{aligned}\\
&\begin{aligned}
\cfrac{\partial^2 \varepsilon\{risk\}}{\partial b^2}=&
0.0234375 b \exp\auxdelta \\
&+7500\auxgamma\exp\auxdelta
+1875\auxalpha \\
&- 0.001875\auxbeta\auxgamma 
b \exp\auxdelta
\end{aligned}
\end{align} 
onde $\auxdelta$ tamb�m � uma vari�vel auxiliar dada por:
\begin{equation}
\auxdelta = {\left(-\frac{b^2}{8\times10^6}\right)}
\end{equation}

Para a execu��o do Exerc�cio 6.44, os C�digos \ref{code:ehl644c1},
\ref{code:ehl644c2} e \ref{code:ehl644c3} foram implemantados em Matlab da
seguinte maneira:
\lstinputlisting[caption={Problema proposto},
label={code:ehl644c1}]{trab2/EHL-6-44/ObjFunc.m} 

\clearpage
\lstinputlisting[caption={Gradiente do problema proposto},
label={code:ehl644c2}]{trab2/EHL-6-44/ObjFuncGrad.m} 
\lstinputlisting[caption={Hessiana do problema proposto},
label={code:ehl644c3}]{trab2/EHL-6-44/ObjFuncHessian.m} 
sendo que:
\begin{conditions}
x & $b$
\end{conditions}
\begin{figure}[H]
\center
\begin{tikzpicture}
\begin{axis}[
		xlabel = $b$,
        ylabel = $\varepsilon\{risk\}$,
		axis lines = left,
        enlargelimits=true
    ]
\addplot+[mark=none, ultra thick, blue]
table[meta=eps]
{trab2/EHL-6-44/ObjFunc.dat};
\end{axis}
\end{tikzpicture}
\caption{Comportamento de $\varepsilon\{risk\}$.}
\label{fig:ehl6-44}
\end{figure}

\subsubsection*{Solu��o do Exerc�cio 6.44 utilizando o m�todo de Newton}
Os resultados para utilisando o m�todo de Newton foram obtidos com o
\autoref{code:ehl644script1} 
\lstinputlisting[language=Matlab, caption={Script de execu��o},
label={code:ehl644script1}]{trab2/EHL-6-44/Script1.m} 

Os resultados para cada intera��o:
\VerbatimInput[firstline=2, lastline=8]{trab2/EHL-6-44/Resultado1.txt}

Ponto de m�nimo encontrado e fun��o avaliada no ponto �timo:
\begin{equation}
\renewcommand{\arraystretch}{1.2}
x^* = \left[
\begin{array}{c}
2.1873
\end{array}
\right]
\quad \text{e}\quad
\varepsilon\{risk\}(x^*)=-2.3888\times 10^8
\end{equation}
\VerbatimInput[firstline=16, lastline=23]{trab2/EHL-6-44/Resultado1.txt}
\clearpage

\emph{Output} final do Matlab, confirmando que foi utilizado o m�todo de Newton
\VerbatimInput[firstline=31,
lastline=38]{trab2/EHL-6-44/Resultado1.txt}

Valores de $\nabla \varepsilon\{risk\}$ (\code{grad}), $\nabla^2
\varepsilon\{risk\}$ (\code{hessian}) e\\ $\lambda(\nabla^2
\varepsilon\{risk\})$ (\code{eigHessian}) , confirmando que se trata de um ponto de m�nimo.
\VerbatimInput[firstline=41, lastline=53]{trab2/EHL-6-44/Resultado1.txt}
% \VerbatimInput[numbers=left]{trab2/EHL-6-44/Resultado1.txt}

\clearpage
\subsubsection*{Solu��o do Exerc�cio 6.44 utilizando um m�todo quasi-Newton}
Os resultados para o m�todo de Newton foram obtidos com o 
\autoref{code:ehl644script2} \lstinputlisting[language=Matlab, caption={Script de execu��o}, 
label={code:ehl644script2}]{trab2/EHL-6-44/Script2.m} 

Os resultados para cada intera��o:
\VerbatimInput[firstline=5, lastline=12]{trab2/EHL-6-44/Resultado2.txt}

Ponto de m�nimo encontrado e fun��o avaliada no ponto �timo:
\begin{equation}
\renewcommand{\arraystretch}{1.2}
x^* = \left[
\begin{array}{c}
2.1873
\end{array}
\right]
\quad \text{e}\quad
\varepsilon\{risk\}(x^*)=-2.3888\times 10^8
\end{equation}
\VerbatimInput[firstline=22, lastline=29]{trab2/EHL-6-44/Resultado2.txt}
\clearpage

\emph{Output} final do Matlab, confirmando que foi utilizado um m�todo
quasi-Newton
\VerbatimInput[firstline=37,
lastline=44]{trab2/EHL-6-44/Resultado2.txt}\bigbreak

Valores de $\nabla \varepsilon\{risk\}$ (\code{grad}), $\nabla^2
\varepsilon\{risk\}$ (\code{hessian}) e\\ $\lambda(\nabla^2
\varepsilon\{risk\})$ (\code{eigHessian}) , confirmando que se trata de um
ponto de m�nimo.
\VerbatimInput[firstline=47, lastline=59]{trab2/EHL-6-44/Resultado2.txt}
% \VerbatimInput[numbers=left]{trab2/EHL-6-44/Resultado2.txt} 



% \clearpage
% 
% \subsection{Exerc�cio Antoine - Etanol}
% \boxexercise{Antoine - Etanol}{
% alguma coisa aqui\ldots
% }
% 
% Partindo de\ldots
% 
 
 
\newdimen\bibindent
\setlength\bibindent{1.5em} % identa�ao das referencias
{ % Um lineskip menor neste contexto de referencias
\rhead{} 
\baselineskip 4.3mm 
% Edite o arquivo bib.bib com as suas pr�prias refer�ncias 
\bibliography{bib}
} 
\end{document}
