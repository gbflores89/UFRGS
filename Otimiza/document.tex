
\documentclass[a4paper,11pt]{article}

\usepackage[T1]{fontenc}
\usepackage{fancyhdr}
\pagestyle{fancy}
\usepackage{ae}
\usepackage[latin1]{inputenc}
\usepackage[brazil]{babel}
\usepackage[brazil]{hyperref}
 
\usepackage{graphicx}
\usepackage{url}
\usepackage{lastpage}
\usepackage{multirow}
\usepackage{indentfirst}
\usepackage{amsmath}
\usepackage{siunitx}
\usepackage{booktabs}
\usepackage{pgfplots}
\usepackage{xifthen}

\usepackage{subfloat}
\usepackage{float}
\usepackage{subfig}
\usepackage{varwidth}
\newcommand{\subfigref}[1]{\hyperref[#1]{Figura~\ref*{#1}}}

\sisetup{output-decimal-marker = {,}}
\newcommand{\theauthori}{}
\newcommand{\theauthorii}{}
\newcommand{\theauthoriii}{}
\renewcommand{\title}[1]{\newcommand{\thetitle}{#1}}
\renewcommand{\author}[1]{\newcommand{\theauthor}{#1}}
\newcommand{\authori}[1]{\renewcommand{\theauthori}{#1}}
\newcommand{\authorii}[1]{\renewcommand{\theauthorii}{#1}}
\newcommand{\authoriii}[1]{\renewcommand{\theauthoriii}{#1}}
\newcommand{\class}[1]{\newcommand{\theclass}{#1}}
\newcommand{\classProffessor}[1]{\newcommand{\theclassProffessor}{#1}}
\usepackage{palatino}
\renewcommand{\textsc}[1]{\fontshape{sc} \fontfamily{\sfdefault} \selectfont #1}

% o novo maketitlepage
\renewcommand{\maketitle}{
% Primeira p�gina de t�tulo  
\pagestyle{empty}
\begin{center}  
\textsc \large
Universidade Federal do Rio Grande do Sul \\
Escola de Engenharia \\
Departamento de Engenharia Qu�mica \\
Programa de P�s--Gradua��o em Engenharia Qu�mica
\vfill
\Large \textsc \theclass \\ Prof:~\theclassProffessor
\vfill \vfill
\huge \bfseries \textsc  
\thetitle
\vfill \vfill
\begin{varwidth}[t]{\textwidth}
\Large \bfseries \textsc 
\theauthor\\
\ifthenelse{\equal{\theauthori}{}}{}{\theauthori\\}
\ifthenelse{\equal{\theauthorii}{}}{}{\theauthorii\\}
\ifthenelse{\equal{\theauthoriii}{}}{}{\theauthoriii}
\end{varwidth}
\vfill \vfill
\large \textsc Porto Alegre, RS \\ \today
\end{center}
\clearpage
\setcounter{page}{1}
\pagestyle{fancy}
\headheight 15pt
} % end maketitle 
\begin{document} 
\class{EQP 0026--Otimiza��o de Processos}
\classProffessor{Marcelo Farenzena}
\title{Trabalho 1}
\author{Guilherme Braganholo Fl�res}
% \authori{Aluno2}
% \authorii{Aluno3}
% \authoriii{Aluno4}

\maketitle 
 
\section*{Defini��es importantes} 
\begin{equation*} 
\renewcommand{\arraystretch}{2.5}
\begin{aligned}
f(x)& = \text{polinomio em fun��o do vetor }x  \\ \\
\nabla f(x) &= \left[
\begin{array}{c}
\cfrac{\partial f}{\partial x_1}\\
\cfrac{\partial f}{\partial x_2}\\
\vdots\\
\cfrac{\partial f}{\partial x_n}\\
\end{array}
\right]\\ \\
\nabla^2 f(x) &= \left[
\begin{array}{cccc}
\cfrac{\partial^2 f}{\partial x_1 \partial x_1} 
& \cfrac{\partial^2 f}{\partial x_1 \partial x_2} & \cdots
& \cfrac{\partial^2 f}{\partial x_1 \partial x_n}\\
\cfrac{\partial^2 f}{\partial x_2 \partial x_1} 
& \cfrac{\partial^2 f}{\partial x_2 \partial x_2} & \cdots
& \cfrac{\partial^2 f}{\partial x_2 \partial x_n}\\
\vdots & \vdots & \ddots
& \vdots\\
\cfrac{\partial^2 f}{\partial x_n \partial x_1} 
& \cfrac{\partial^2 f}{\partial x_n \partial x_2} & \cdots
& \cfrac{\partial^2 f}{\partial x_n \partial x_n}\\
\end{array}
\right]
\end{aligned}
\end{equation*}
\clearpage

\section {Convexidade}
Analisar a convexidade das seguintes fun��es objetivo:
\begin{align}
f_1 (x) &= 2x_1 + 3x_2 +6 \label{eq:1f1}\\ 
f_2 (x) &= x_1^3 \label{eq:1f2}\\
f_3 (x) &= x_1^2 + x_1 x_2 + x_2 + 4 \label{eq:1f3}
\end{align}

\subsection*{\autoref{eq:1f1}}
\begin{equation}
f_1 (x) = 2x_1 + 3x_2 +6 \tag{\ref{eq:1f1}}
\end{equation}
Para a \autoref{eq:1f1} temos que o gradiente de $f_1(x)$ � dado por:
\begin{equation}
\nabla f_1(x) = \left[
\begin{array}{c}
2\\
3
\end{array}
\right]
\end{equation}
e para a hessiana temos:
\begin{equation}
\nabla^2 f_1(x) = \left[
\begin{array}{cc}
0& 0\\
0& 0
\end{array}
\right]
\end{equation}

Calculando os autovalores da matriz $\nabla^2 f_1(x)$ temos para $\forall x$:
\begin{itemize}
  \item $\lambda_1 = 0$
  \item $\lambda_2 = 0$
\end{itemize}
portanto temos uma fun��o convexa, neste caso um plano, dado que a fun��o �
linear.


\subsection*{\autoref{eq:1f2}}
\begin{equation}
f_2 (x) = x_1^3 \tag{\ref{eq:1f2}}
\end{equation}
Para a Equa��o \autoref{eq:1f2} temos que o gradiente de $f_2(x)$ � dado por:
\begin{equation}
\nabla f_2(x) = \left[
\begin{array}{c}
3x_1^2
\end{array}
\right]
\end{equation}
e para a hessiana temos:
\begin{equation}
\nabla^2 f_2(x) = \left[
\begin{array}{c}
6x_1
\end{array}
\right]
\end{equation}

Calculando os autovalores da matriz $\nabla^2 f_2(x)$ temos:
\begin{itemize}
  \item $\lambda_1 = \left\{
\begin{array}{ccc}
\geq 0& \rm{se}& x_1 \geq 0\\
\leq 0& \rm{se}& x_1 \leq 0
\end{array}
\right.$
\end{itemize}
portanto temos uma fun��o convexa para valores positivos de $x_1$ e n�o convexa
para valores de $x_1$ menores que $0$.
\begin{figure}[h]
\centering
\def\FunctionF(#1){(#1)^3}
\def\FunctionFd(#1){3*(#1)^2}
\def\FunctionFdd(#1){6*(#1)}
\subfloat[$f_1(x)$]{
\begin{tikzpicture}[scale=0.8]
\begin{axis}[
        axis y line=center,
        axis x line=middle, 
        axis on top=true,
    ]
    \addplot [domain=-2:2, samples=51, mark=none, ultra thick, blue] {\FunctionF(x)};
    \node [right, blue] at (axis cs: 0.5,4.5) {$f_2(x)$};
\end{axis}
\end{tikzpicture}}
\hfill
\subfloat[$\nabla f_1(x)$]{
\begin{tikzpicture}[scale=0.8]
\begin{axis}[
        axis y line=center,
        axis x line=middle, 
        axis on top=true,
    ]
    \addplot [domain=-2:2, samples=51, mark=none, ultra thick, orange]
    {\FunctionFd(x)};
	\addplot [domain=-2:2, samples=51, mark=none, ultra thick, red]
	{\FunctionFdd(x)};
\node [right, orange] at (axis cs: 0.5,-5) {$\nabla f_2(x)$};
\node [right, red] at (axis cs: 0.5,-7) {$\nabla^2 f_2(x)$};
\end{axis}
\end{tikzpicture}}
\caption{$f_2(x) = x_1^3$.}
\label{fig:1f2}
\end{figure}



\subsection*{\autoref{eq:1f3}}
\begin{equation}
f_3 (x) = x_1^2 + x_1 x_2 + x_2 + 4  \tag{\ref{eq:1f3}}
\end{equation}
Para a Equa��o \autoref{eq:1f3} temos que o gradiente de $f_3(x)$ � dado por:
\begin{equation}
\nabla f_3(x) = \left[
\begin{array}{c}
2x_1 + x_2 \\
x_1 + 2
\end{array}
\right]
\end{equation}
e para a hessiana temos:
\begin{equation}
\nabla^2 f_3(x) = \left[
\begin{array}{cc}
2 & 1\\
1 & 0
\end{array}
\right]
\end{equation}

Calculando os autovalores da matriz $\nabla^2 f_3(x)$ temos para $\forall x$:
\begin{itemize}
  \item $\lambda_1 = 1 + \sqrt{2} = 2.41$
  \item $\lambda_2 = 1 - \sqrt{2} = -0.41$
\end{itemize}

Sendo esta matriz n�o definida, n�o � uma fun��o convexa.

\clearpage
\section {Pontos de M�nimo e M�ximo}
Analisar pontos de m�ximo e m�nimo das seguintes fun��es:
\begin{align}
f_1 (x) &= |x| \label{eq:2f1}\\ 
f_2 (x) &= -x_1^4 + x_1^3 +20 \label{eq:2f2}\\
f_3 (x) &= x_1^2 + 2x_1 + 3x_2^2 + 6x_2 + 2 \label{eq:2f3}
\end{align}

\subsection*{\autoref{eq:2f1}}
\begin{equation}
f_1 (x) = |x| \tag{\ref{eq:2f1}} 
\end{equation}
Para a Equa��o \autoref{eq:2f1} temos que o gradiente de $f_1(x)$ � dado por:
\begin{equation}
\nabla f_1(x) = \left[
\begin{array}{cc}
1 &\quad \text{se } x> 0\\
-1 &\quad \text{se } x < 0
\end{array}
\right]
\end{equation}

Aplicando em $x_1 = 0$ temos $f_1(x) = 0$ e para $\forall$ valores de $x$
aplicados em $f_1(x)$ s�o obtidos valores maiores que $0$, como podemos observar
na \autoref{fig:2f1}:
\begin{figure}[h]
\centering
\def\FunctionF(#1){abs(#1)}
\def\FunctionFd(#1){1}
\def\FunctionFdd(#1){-1}
\subfloat[$f_1(x)$]{
\begin{tikzpicture}[scale=0.8]
\begin{axis}[
        axis y line=center,
        axis x line=middle, 
        axis on top=true,
        xmin=-5,
        xmax=5,
        ymin=0,
        ymax=5,
    ]
    \addplot [domain=-5:5, samples=51, mark=none, ultra thick, blue] {\FunctionF(x)};
    \node [right, blue] at (axis cs: 0.5,4.5) {$f_1(x)$};
\end{axis}
\end{tikzpicture}}
\hfill
\subfloat[$\nabla f_1(x)$]{
\begin{tikzpicture}[scale=0.8]
\begin{axis}[
        axis y line=center,
        axis x line=middle, 
        axis on top=true,
        xmin=-5,
        xmax=5,
        ymin=-1.2,
        ymax=1.2,
    ]
    \addplot [domain=0:5, samples=51, mark=none, ultra thick, orange]
    {\FunctionFd(x)};
	\addplot [domain=-5:0, samples=51, mark=none, ultra thick, orange]
	{\FunctionFdd(x)}; 
\node [right, orange] at (axis cs: 0.5,0.5) {$\nabla f_1(x)$};
\end{axis}
\end{tikzpicture}}
\caption{$f_1(x) = |x|$.}
\label{fig:2f1}
\end{figure}

Portanto podemos concluir que temos um ponto de m�nimo, apesar do gradiente de
$f_1(x)$ n�o ser cont�nuo e portanto nao ser poss�vel o calculo da hessiana de
$f_1(x)$.

\subsection*{\autoref{eq:2f2}}
\begin{equation}
f_2 (x) = -x_1^4 + x_1^3 +20 \tag{\ref{eq:2f2}} 
\end{equation}
Para a Equa��o \autoref{eq:2f2} temos que o gradiente de $f_2(x)$ � dado por:
\begin{equation}
\nabla f_2(x) = \left[
\begin{array}{c}
-4x_1^3 + 3x_1^2
\end{array}
\right]
\end{equation}
Para um ponto de m�nimo ou m�ximo devemos obter $\nabla f_2(x) = 0$, o que
ocorre quando:
\begin{itemize}
  \item $x_1 = 0$
  \item $x_1 = \cfrac{3}{4}$
\end{itemize}
portanto devemos analizar o valor da hessiana nestes dois pontos, como �
poss�vel observar na \autoref{fig:2f2}.

\begin{figure}[h]
\centering
\def\FunctionF(#1){-(#1)^4 + (#1)^3 +20}
\def\FunctionFd(#1){-4*(#1)^3 + 3*(#1)^2}
\def\FunctionFdd(#1){-12*(#1)^2 + 6*(#1)}

\subfloat[$f_2(x)$]
{\begin{tikzpicture}[scale=0.8]
\begin{axis}[
        axis y line=center,
        axis x line=middle, 
        axis on top=true,
        xmin=-1,
        xmax=1.5,
        ymin=19,
        ymax=21,
    ]
    \addplot [domain=-2:2.5, samples=100, 
    mark=none, ultra thick, blue]{\FunctionF(x)};
\node [right, blue] at (axis cs: 0.5,20.5) {$f_2(x)$};
\end{axis}
\end{tikzpicture}}
\subfloat[$\nabla f_2(x)$ e $\nabla^2 f_2(x)$]
{\begin{tikzpicture}[scale=0.8]
\begin{axis}[
        axis y line=center,
        axis x line=middle, 
        axis on top=true,
        xmin=-1,
        xmax=1,
        ymin=-1,
        ymax=2,
    ]
    \addplot [domain=-2:2.5, samples=100, 
    mark=none, ultra thick, orange]{\FunctionFd(x)};
    \addplot [domain=-2:2.5, samples=100, 
    mark=none, ultra thick, red]{\FunctionFdd(x)};
\node [right, orange] at (axis cs: 0.1,1.5) {$\nabla f_2(x)$};
\node [right, red] at (axis cs: 0.1,1.0) {$\nabla^2 f_2(x)$};
\end{axis}
\end{tikzpicture}}
\caption{$f_2 (x) = -x_1^4 + x_1^3 +20$.}
\label{fig:2f2}
\end{figure}

A hessiana de $f_2(x)$ � dada por:
\begin{equation}
\nabla^2 f_2(x) = \left[
\begin{array}{c}
-12x_1^2 + 6x_1
\end{array}
\right]
\end{equation}

Aplicada em $x_1 = 0$ em $\nabla f_2(x)$ temos:
\begin{equation}
\nabla^2 f_2(0) = \left[
\begin{array}{c}
0
\end{array}
\right]
\end{equation}
onde n�o podemos concluir se � um ponto de m�ximo ou m�nimo, pois � uma matriz
indefinida, o que leva a creer que seja um ponto de inflex�o. Isto pode ser
observando na \autoref{fig:2f2} onde � apresentado o comportamento da fun��o
$f_2(x)$.

Aplicada em $x_1 = \cfrac{3}{4}$ em $\nabla f_2(x)$ temos:
\begin{equation}
\nabla^2 f_2 \left(\frac{3}{4}\right) = \left[
\begin{array}{c}
-2.25
\end{array}
\right]
\end{equation}
onde podemos concluir diretamente, pelo autovalor �nico $(\lambda_1 = -2.25)$,
que se trata de um ponto de m�ximo, para este caso.

\subsection*{\autoref{eq:2f3}}
\begin{equation}
f_3 (x) = x_1^2 + 2x_1 + 3x_2^2 + 6x_2 + 2 \tag{\ref{eq:2f3}} 
\end{equation}
Para este �ltimo caso, temos $\nabla f_3(x)$ dado por:
\begin{equation}
\nabla f_3(x) = \left[
\begin{array}{c}
2x_1 + 2\\
6x_2 + 6
\end{array}
\right]
\end{equation}

Para a condi��o necess�ria de m�ximo ou minimos devemos ter $\nabla f_3(x) =
0$, isto ocorre apenas em $x = [-1, -1]^{\rm{T}}$. Partindo da hessiana de
$f_3(x)$ que � igual �:
\begin{equation}
\nabla^2 f_3(x) = \left[
\begin{array}{cc}
2 & 0\\
0& 6
\end{array}
\right]
\end{equation}
e aplicando em $x = [-1, -1]^{\rm{T}}$ temos os uma matriz positiva definida,
com os seguintes autovalores:
\begin{itemize}
  \item $\lambda_1 = 6$
  \item $\lambda_2 = 2$
\end{itemize}
portanto, um ponto de m�nimo.

  


%\bibliographystyle{finplain}
% \bibliographystyle{plain}
% \bibliography{}

\end{document}
