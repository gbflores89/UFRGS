
\section {Programa��o Linear}
\boxexercise{4}
{  
Uma refinaria de petr�leo, na blendagem da gasolina, mistura quatro
constituintes (cortes) do petr�leo em tr�s tipos de gasolina (A, B e C). O
objetivo � simples - misturar adequadamente para maximizar o lucro. 
A disponibilidade e o custo de cada um dos 4 constituintes � dado por: 
\begin{table}[H]
\renewcommand{\arraystretch}{1.1}
\footnotesize
\begin{center}
\begin{tabular}{S[table-format=1.0]S[table-format=4.0]S[table-format=2.2,round-mode=places,round-precision=2]}
\toprule
{Constituent}&{\makecell{Maximum quantity \\ avaliable
\\ (bbl/day)}}&{\makecell{Cost \\ per barrel (\$)}}\\
\midrule
1&3000&13.\\
2&2000&15.3\\
3&4000&14.6\\
4&1000&14.9\\
\bottomrule
\end{tabular}
\end{center}
\end{table} 

Os requisitos de cada gasolina s�o: 
\begin{table}[H]
\renewcommand{\arraystretch}{1.1}
\footnotesize
\begin{center}
\begin{tabular}{clS[table-format=2.2,round-mode=places,round-precision=2]}
\toprule
{Grade}&\multicolumn{1}{c}{Specification}&{\makecell{Selling price \\ per
barrel (\$)}}\\
\midrule
A&\makecell[l]{Not more than $15\%$ of 1 
\\ Not less than $40\%$ of 2
\\ Not more than $50\%$ of 3} &16.2\\
\hline
B&\makecell[l]{Not more than $10\%$ of 1
\\ Not less than $10\%$ of 2}&15.75\\
\hline
C&\makecell[l]{Not more than $20\%$ of 1}&15.3\\
\bottomrule
\end{tabular}
\end{center}
\end{table} 
}

